\documentclass[12pt]{scrartcl}


\setlength{\parindent}{0pt}
\setlength{\parskip}{.25cm}

\usepackage{graphicx}

\usepackage{xcolor}

\definecolor{darkred}{rgb}{0.5,0,0}
\definecolor{darkgreen}{rgb}{0,0.5,0}
\usepackage{hyperref}
\hypersetup{
  letterpaper,
  colorlinks,
  linkcolor=red,
  citecolor=darkgreen,
  menucolor=darkred,
  urlcolor=blue,
  pdfpagemode=none,
  pdfkeywords={}
}

\definecolor{MyDarkBlue}{rgb}{0,0.08,0.45}
\definecolor{MyDarkRed}{rgb}{0.45,0.08,0}
\definecolor{MyDarkGreen}{rgb}{0.08,0.45,0.08}

\definecolor{mintedBackground}{rgb}{0.95,0.95,0.95}
\definecolor{mintedInlineBackground}{rgb}{.90,.90,1}

%\usepackage{newfloat}
\usepackage[newfloat=true]{minted}
\setminted{mathescape,
               linenos,
               autogobble,
               frame=none,
               framesep=2mm,
               framerule=0.4pt,
               %label=foo,
               xleftmargin=2em,
               xrightmargin=0em,
               startinline=true,  %PHP only, allow it to omit the PHP Tags *** with this option, variables using dollar sign in comments are treated as latex math
               numbersep=10pt, %gap between line numbers and start of line
               style=default, %syntax highlighting style, default is "default"
               			    %gallery: http://help.farbox.com/pygments.html
			    	    %list available: pygmentize -L styles
               bgcolor=mintedBackground} %prevents breaking across pages
               
\setmintedinline{bgcolor={mintedBackground}}
\setminted[text]{bgcolor={mintedBackground},linenos=false,autogobble,xleftmargin=1em}
%\setminted[php]{bgcolor=mintedBackgroundPHP} %startinline=True}
\SetupFloatingEnvironment{listing}{name=Code Sample}
\SetupFloatingEnvironment{listing}{listname=List of Code Samples}



\title{CSCE 155 - C}
\subtitle{Lab 13.0 - Searching \& Sorting}
\author{Cole Peterson}
\date{~}

\begin{document}

\maketitle

\section*{Prior to Lab}

Before attending this lab:
\begin{enumerate}
  \item Read and familiarize yourself with this handout.
  \item Review notes on Search \& Sorting
\end{enumerate}

Some additional resources that may help with this lab:
\begin{itemize}
  \item Wikipedia Article on Selection Sort: \\
  	\url{http://en.wikipedia.org/wiki/Selection_sort}
  \item Wikipedia Article on Quicksort: \\
  	\url{http://en.wikipedia.org/wiki/Quicksort}
  \item Online manual page for \mintinline{c}{qsort}: \\
  	\url{http://man7.org/linux/man-pages/man3/qsort.3.html}
  \item Online manual page for \mintinline{c}{bsearch}: \\
  	\url{http://man7.org/linux/man-pages/man3/bsearch.3.html}
\end{itemize}

\section*{Peer Programming Pair-Up}

To encourage collaboration and a team environment, labs will be
structured in a \emph{pair programming} setup.  At the start of
each lab, you will be randomly paired up with another student 
(conflicts such as absences will be dealt with by the lab instructor).
One of you will be designated the \emph{driver} and the other
the \emph{navigator}.  

The navigator will be responsible for reading the instructions and
telling the driver what to do next.  The driver will be in charge of the
keyboard and workstation.  Both driver and navigator are responsible
for suggesting fixes and solutions together.  Neither the navigator
nor the driver is ``in charge.''  Beyond your immediate pairing, you
are encouraged to help and interact and with other pairs in the lab.

Each week you should alternate: if you were a driver last week, 
be a navigator next, etc.  Resolve any issues (you were both drivers
last week) within your pair.  Ask the lab instructor to resolve issues
only when you cannot come to a consensus.  

Because of the peer programming setup of labs, it is absolutely 
essential that you complete any pre-lab activities and familiarize
yourself with the handouts prior to coming to lab.  Failure to do
so will negatively impact your ability to collaborate and work with 
others which may mean that you will not be able to complete the
lab.  

\section{Lab Objectives \& Topics}
At the end of this lab you should be familiar with the following
\begin{itemize}
  \item Understand basic searching and sorting algorithms
  \item Understand how comparator functions work and their purpose
  \item How to leverage standard search and sort algorithms built into a framework
\end{itemize}

\section{Background}

Recursion has been utilized in several searching and sorting 
algorithms. In particular, quicksort and binary search both can 
be implemented using recursive functions.  The quicksort 
algorithm works by choosing a pivot element and dividing a 
list into two sub-lists consisting of elements smaller than the 
pivot and elements that are larger than the pivot (ties can be 
broken either way).  A recursive quicksort function can then 
be recursively called on each sub-list until a base case is 
reached: when the list to be sorted is of size 1 or 0 which, 
by definition, is already sorted.

Binary search can also be implemented in a recursive manner.  
Given a sorted array and a key to be searched for in the array, 
we can examine the middle element in the array.  If the key is 
less than the middle element, we can recursively call the binary 
search function on the sub-list of all elements to the left of the 
middle element.  If the key is greater than the middle element, 
we recursively call the binary search function on the sub-list of 
all elements to the right of the middle element.

Searching and sorting are two solved problems.  That is, most 
languages and frameworks offer built-in functionality or 
standard implementations in their standard libraries to facilitate 
searching and sorting through collections.  These implementations 
have been optimized, debugged and tested beyond anything that 
a single person could ever hope to accomplish.  Thus, there is 
rarely ever a good justification for implementing your own 
searching and sorting methods.  Instead, it is far more important 
to understand how to leverage the framework and utilize the 
functionality that it already provides.  

\section{Activities}

The activities in this lab will involve the same baseball data as 
used in a prior lab.  A complete framework has been provided 
to you to load data on the 2011 National League teams.  There 
is more data this time around and we have defined a structure 
to encapsulate team data as well as several functions to process 
the input file and construct instances of the \mintinline{c}{Team} 
structures for you.

Clone the project code for this lab from GitHub using the following 
URL: \url{https://github.com/clptrsn/CSCE155-C-Lab13}.

\subsection{Sorting the Wrong Way}

In this activity, you are to implement a selection sort algorithm to 
sort an array of \mintinline{c}{Team} structures.  Refer to lecture notes, 
your text, or any online source on how to implement the selection sort 
algorithm.  The order by which you will sort the array will be according 
to the total team payroll (in dollars) in increasing order.

\subsubsection*{Instructions}

\begin{enumerate}
  \item Familiarize yourself with the \mintinline{c}{Team} structure and 
	the functions provided to you (the \mintinline{c}{main} program 
	automatically loads the data from the data file and provides an 
	array of teams).  
  \item Implement the \mintinline{c}{selectionSortTeamsByPayroll} 
	function in the \mintinline{text}{mlb.c} file as specified
  \item Compile and run your program (use make which produces an 
	executable called \mintinline{text}{mlbDriver})
\end{enumerate}

\subsection{Slightly Better Sorting}

The \mintinline{c}{Team} structure has many different data fields; 
wins, losses (and win percentage), team name, city, state, payroll, 
average salary.  Say that we wanted to re-sort teams according to 
one of these fields in either ascending or descending order (or even 
some combination of fields? state/city for example).  Doing it the 
wrong way as above we would need to implement no less than 
16 different sort functions!

Most of the code would be the same though; only the criteria on 
which we update the index of the minimum element would differ.  
Instead, it would be better to implement one sorting function and 
make it configurable: provide a means by which the function can 
determine the relative ordering of any two elements.  The way of 
doing this in C is to define a comparator function.

Comparator functions are simple: they take two arguments 
$a, b$ (specified by generic void pointers) and return:
\begin{itemize}
  \item A negative value if $a < b$ 
  \item Zero if $a$ is equal to $b$
  \item A positive value if $a > b$
\end{itemize}
Several examples have been provided for comparing \mintinline{c}{Team} 
structures based on several different criteria.  The usual pattern 
is to cast the arguments to the expected types, then to examine 
the relevant field(s) and return a result that orders the two teams 
appropriately.

\subsubsection*{Instructions}

\begin{enumerate}
  \item Implement the \mintinline{c}{selectionSortTeams} function 
	by using the code in Activity 1 with appropriate modifications 
	(use the provided \mintinline{c}{compar} function to find the 
	minimum element each time)
  \item Look to the comparator functions provided to you and to the 
	bubble sort algorithm example on how you might use a comparator 
	function
  \item Implement your own comparator function that orders \mintinline{c}{Team}s 
	according to the total payroll in decreasing order.
  \item Use your comparator in the \mintinline{text}{mlbDemo.c} program 
	to call your \mintinline{c}{selectionSortTeams} function.
\end{enumerate}
	
\subsection{Sorting the Right Way}

The better way of doing this is to leverage the standard C library's 
\mintinline{c}{qsort} sorting function.  The signature of this function 
is as follows:

\begin{minted}{c}
void qsort(void *base, size_t nmemb, size_t size, 
	int(*compar)(const void *, const void *));
\end{minted}

where
\begin{itemize}
  \item \mintinline{c}{base} is the array to be sorted
  \item \mintinline{c}{nmemb} is the number of elements in the array  
  \item \mintinline{c}{size} is the size of each element (use \mintinline{c}{sizeof()})
  \item \mintinline{c}{compar} is a comparator function pointer
\end{itemize}

\subsubsection*{Instructions}

\begin{enumerate}
  \item Examine the source files and observe how comparator functions 
	are implemented and how the \mintinline{c}{qsort} function is called.
  \item Implement your own comparator function that orders \mintinline{c}{Team}s 
	according to win percentage in increasing order.
  \item Use your function in the  \mintinline{c}{main} program to re-sort the 
	array and print out the results.
\end{enumerate}
	
\subsection{Searching}

The standard C library offers several search functions.  The two that we 
will focus on are as follows.
\begin{itemize}
  \item \mintinline{c}{lfind} - a linear search implementation that does not 
  	require the array to be sorted.  The function works by iterating through 
	the array and calling your comparator function on the provided key.  
	It returns the first instance such that the comparator returns zero (equal).
  \item \mintinline{c}{bsearch} - a binary search implementation that requires 
  	the array to be sorted according to the same comparator used to search.
\end{itemize}
	
Both functions require a comparator function (as used in sorting) and a 
key upon which to search.  A key is a ``dummy'' instance of the same type 
as the array that contains values of fields that you?re searching for.

\subsubsection*{Instructions}

\begin{enumerate}
  \item Examine the searching code segment in the \mintinline{text}{mlbDriver.c} 
	and understand how each function is called.  
  \item Answer the questions in your worksheet regarding this code segment.
  \item Based on your observations add code to search the array for the 
	team representing the Chicago Cubs.
  \begin{enumerate}
    \item Create a dummy \mintinline{c}{Team} key for the Cubs by 
    calling the \mintinline{c}{createTeam} function using empty strings and 
    zero values except for the team name (which should be \mintinline{c}{"Cubs"}).
    \item Sort the array by team name by calling \mintinline{c}{qsort} using 
    the appropriate comparator function.
    \item Call the \mintinline{c}{bsearch} function using your key and the 
    appropriate comparator function to find the \mintinline{c}{Team} representing 
    the Chicago Cubs.
    \item Print out the team by using the \mintinline{c}{printTeam} function.
  \end{enumerate}
  \item Answer the questions in your worksheet and demonstrate your 
    working program to a lab instructor.
\end{enumerate}



\section{Handin/Grader Instructions}

\begin{enumerate}
  \item If you are performing the lab asynchronously, follow these instructions to hand in your code.
  \item Hand in your \mintinline{text}{mlb.c}, \mintinline{text}{mlb.h}, and \mintinline{text}{mlbDriver.c} source file by pointing your browser to:
  	\url{https://cse-apps.unl.edu/handin}
	login/password.
  \item Grade yourself by pointing your browser to
  	\url{https://cse.unl.edu/~cse155e/grade/}
\end{enumerate}

\section{Advanced Activity (Optional)}

Selection sort is a quadratic sorting algorithm, thus doubling the input size 
($n$ elements to $2n$ elements) leads to a blowup in its execution time 
by a factor of 4.  Quick sort requires only $n\log(n)$ operations on average, 
so doubling the input size would only lead (roughly) to a blowup in execution 
time by a factor of about 2.  Verify this theoretical analysis by setting up an 
experiment to time each sorting algorithm on various input sizes of randomly 
generated integer arrays.  Make use of the time library?s functions to time 
the execution of your function calls.  
	
\end{document}
